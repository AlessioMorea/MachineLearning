\documentclass[12pt,english, openany]{article}		%definire grandezza testo, foglio e tipo di testo.
\usepackage{pstricks,graphicx,amsmath,bbm,mathrsfs,amssymb,psfrag,pifont,times,mathptmx}
\usepackage[utf8]{inputenc}
\usepackage{hyperref}
\usepackage[english]{babel}
\usepackage{color}
\usepackage{ulem}

% Prevents LaTeX from filling out a page to the bottom
\raggedbottom


\def\*#1{\boldsymbol{#1}}
\def\+#1{\mathcal{#1}}
\def\b#1{\mathbb{#1}}


\begin{document}
\title{Machine learning con applicazioni}
\author{Alessio~Morea}

\maketitle
%%%



\begin{abstract}
The aim of this project is to train neural networks for the classification of chess pieces.
\end{abstract}
\section{Dataset}
The dataset \footnote{The dataset can be found at the link \href{https://www.kaggle.com/datasets/mr11261/chess-squares-from-chess-diagrams}{https://www.kaggle.com/datasets/mr11261/chess-squares-from-chess-diagrams}}
 is composed by $6400$ rows and $1025$ columns.
\begin{itemize}
\item Each row represents an image of a chess piece with the respective square color in greyscale;
\item the first column is the 
\item 
\end{itemize}

\cite{10.5555/2188385.2188395}




\bibliographystyle{apsrev4-2}
\bibliography{biblio}% Produces the bibliography via BibTeX.
\end{document}